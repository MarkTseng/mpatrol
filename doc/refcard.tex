% mpatrol
% A library for controlling and tracing dynamic memory allocations.
% Copyright (C) 1997-2008 Graeme S. Roy <graemeroy@users.sourceforge.net>
%
% This program is free software: you can redistribute it and/or modify it
% under the terms of the GNU Lesser General Public License as published by
% the Free Software Foundation, either version 3 of the License, or (at
% your option) any later version.
%
% This program is distributed in the hope that it will be useful, but
% WITHOUT ANY WARRANTY; without even the implied warranty of
% MERCHANTABILITY or FITNESS FOR A PARTICULAR PURPOSE.  See the GNU Lesser
% General Public License for more details.
%
% You should have received a copy of the GNU Lesser General Public License
% along with this program.  If not, see <http://www.gnu.org/licenses/>.

% LaTeX reference card for mpatrol

% $Id$

\documentclass[a4paper,landscape,final]{article}

\usepackage{multicol}

% Adapt the page dimensions to the paper size.

\setlength{\textwidth}{\paperwidth}
\addtolength{\textwidth}{-1in}
\setlength{\textheight}{\paperheight}
\addtolength{\textheight}{-1in}
\setlength{\headheight}{0in}
\setlength{\headsep}{0in}
\setlength{\footskip}{0in}
\setlength{\parindent}{0in}
\setlength{\columnsep}{.5in}
\setlength{\oddsidemargin}{-.5in}
\setlength{\evensidemargin}{-.5in}
\setlength{\topmargin}{-.5in}

% Define new commands for formatting the headings and options.

\newcommand{\heading}[1]{\textbf{\normalsize #1}}
\newcommand{\command}[1]{\textbf{#1}}
\newcommand{\function}[1]{\texttt{#1()}}
\newcommand{\variable}[1]{\texttt{#1}}
\newcommand{\error}[1]{\texttt{#1}}
\newcommand{\flag}[1]{\textbf{---#1}}
\newcommand{\flagpar}[2]{\flag{#1} \texttt{<}\textit{#2}\texttt{>}}
\newcommand{\option}[1]{\texttt{#1}}
\newcommand{\optionarg}[2]{\option{#1}\texttt{=<#2>}}
\newcommand{\optionpar}[2]{\option{#1}\texttt{=<}\textit{#2}\texttt{>}}

\begin{document}

\pagestyle{empty}

\footnotesize

\begin{multicols}{3}{\textbf{\Large mpatrol reference card}}

\vskip 12pt

The mpatrol library can read certain options at run-time from an environment
variable called \texttt{MPATROL\_OPTIONS}.  This variable must contain one or
more valid option keywords from the list below and must be no longer than 1024
characters in length.  If \texttt{MPATROL\_OPTIONS} is unset or empty then the
default settings will be used.

The \texttt{LOGDIR}, \texttt{PROFDIR} and \texttt{TRACEDIR} environment
variables are also read in order to determine where the log file, profiling
output file and tracing output file should go.  Note that if they are set then
the default filenames for the log file, profiling output file and tracing output
file will also be changed.

\vskip 12pt
\heading{Library behaviour}
\vskip 6pt

\begin{description}
\item[\option{HELP}]
Displays a quick-reference option summary.
\item[\optionpar{PROGFILE}{string}]
Specifies an alternative filename with which to locate the executable file
containing the program's symbols.
\item[\optionpar{CHECK}{unsigned range}]
Specifies a range of allocation indices at which to check the integrity of free
memory and overflow buffers.
\item[\option{EDIT}]
Specifies that a text editor should be invoked to edit any relevant source files
that are associated with any warnings or errors when they occur.
\item[\option{LIST}]
Specifies that a context listing should be shown for any relevant source files
that are associated with any warnings or errors when they occur.
\item[\optionpar{DEFALIGN}{unsigned integer}]
Specifies the default alignment for general-purpose memory allocations, which
must be a power of two.
\item[\option{NOPROTECT}]
Specifies that the mpatrol library's internal data structures should not be made
read-only after every memory allocation, reallocation or deallocation.
\item[\option{SAFESIGNALS}]
Instructs the library to save and replace certain signal handlers during the
execution of library code and to restore them afterwards.
\item[\option{CHECKFORK}]
Checks at every call to see if the process has been forked in case new log,
profiling and tracing output files need to be started.
\item[\option{USEMMAP}]
Specifies that the library should use \function{mmap} instead of \function{sbrk}
to allocate user memory on UNIX platforms.
\end{description}

\vskip 12pt
\heading{Logging and tracing}
\vskip 6pt

\begin{description}
\item[\optionpar{LOGFILE}{string}]
Specifies an alternative file in which to place all diagnostics from the mpatrol
library.
\item[\option{LOGALLOCS}]
Specifies that all memory allocations are to be logged and sent to the log file.
\item[\option{LOGREALLOCS}]
Specifies that all memory reallocations are to be logged and sent to the log
file.
\item[\option{LOGFREES}]
Specifies that all memory deallocations are to be logged and sent to the log
file.
\item[\option{LOGMEMORY}]
Specifies that all memory operations are to be logged and sent to the log file.
\item[\option{LOGALL}]
Equivalent to the \option{LOGALLOCS}, \option{LOGREALLOCS}, \option{LOGFREES}
and \option{LOGMEMORY} options specified together.
\item[\option{TRACE}]
Specifies that all memory allocations are to be traced and sent to the tracing
output file.
\item[\optionpar{TRACEFILE}{string}]
Specifies an alternative file in which to place all memory allocation tracing
information from the mpatrol library.
\item[\option{LEAKTABLE}]
Specifies that the leak table should be automatically used and a leak table
summary should be displayed at the end of program execution.
\item[\option{SHOWFREE}]
Specifies that a summary of all of the free memory blocks should be displayed at
the end of program execution.
\item[\option{SHOWFREED}]
Specifies that a summary of all of the freed memory allocations should be
displayed at the end of program execution.
\item[\option{SHOWUNFREED}]
Specifies that a summary of all of the unfreed memory allocations should be
displayed at the end of program execution.
\item[\option{SHOWMAP}]
Specifies that a memory map of the entire heap should be displayed at the end of
program execution.
\item[\option{SHOWSYMBOLS}]
Specifies that a summary of all of the function symbols read from the program's
executable file should be displayed at the end of program execution.
\item[\option{SHOWALL}]
Equivalent to the \option{SHOWFREE}, \option{SHOWFREED}, \option{SHOWUNFREED},
\option{SHOWMAP} and \option{SHOWSYMBOLS} options specified together.
\item[\option{USEDEBUG}]
Specifies that any debugging information in the executable file should be used
to obtain additional source-level information.
\end{description}

\vskip 12pt
\heading{General errors}
\vskip 6pt

\begin{description}
\item[\option{CHECKALLOCS}]
Checks that no attempt is made to allocate a block of memory of size zero.
\item[\option{CHECKREALLOCS}]
Checks that no attempt is made to reallocate a \texttt{NULL} pointer or resize
an existing block of memory to size zero.
\item[\option{CHECKFREES}]
Checks that no attempt is made to deallocate a \texttt{NULL} pointer.
\item[\option{CHECKMEMORY}]
Checks that no attempt is made to perform a zero-length memory operation or a
memory operation on a \texttt{NULL} pointer.
\item[\option{CHECKALL}]
Equivalent to the \option{CHECKALLOCS}, \option{CHECKREALLOCS},
\option{CHECKFREES} and \option{CHECKMEMORY} options specified together.
\item[\optionpar{ALLOCBYTE}{unsigned integer}]
Specifies an 8-bit byte pattern with which to prefill newly-allocated memory.
\item[\optionpar{FREEBYTE}{unsigned integer}]
Specifies an 8-bit byte pattern with which to prefill newly-freed memory.
\item[\optionpar{NOFREE}{unsigned integer}]
Specifies that a number of recently-freed memory allocations should be prevented
from being returned to the free memory pool.
\item[\option{PRESERVE}]
Specifies that any reallocated or freed memory allocations should preserve their
original contents.
\end{description}

\vskip 12pt
\heading{Overwrites and underwrites}
\vskip 6pt

\begin{description}
\item[\optionpar{OFLOWSIZE}{unsigned integer}]
Specifies the size in bytes to use for all overflow buffers, which must be a
power of two.
\item[\optionpar{OFLOWBYTE}{unsigned integer}]
Specifies an 8-bit byte pattern with which to fill the overflow buffers of all
memory allocations.
\item[\option{OFLOWWATCH}]
Specifies that watch point areas should be used for overflow buffers rather than
filling with the overflow byte.
\item[\optionarg{PAGEALLOC}{LOWER|UPPER}]
Specifies that each individual memory allocation should occupy at least one page
of virtual memory and should be placed at the lowest or highest point within
these pages.
\item[\option{ALLOWOFLOW}]
Specifies that a warning rather than an error should be produced if any memory
operation function overflows the boundaries of a memory allocation, and that
the operation should still be performed.
\end{description}

\vskip 12pt
\heading{Using with a debugger}
\vskip 6pt

\begin{description}
\item[\optionpar{ALLOCSTOP}{unsigned integer}]
Specifies an allocation index at which to stop the program when it is being
allocated.
\item[\optionpar{REALLOCSTOP}{unsigned integer}]
Specifies an allocation index at which to stop the program when a memory
allocation is being reallocated.
\item[\optionpar{FREESTOP}{unsigned integer}]
Specifies an allocation index at which to stop the program when it is being
freed.
\end{description}

\vskip 12pt
\heading{Testing}
\vskip 6pt

\begin{description}
\item[\optionpar{LIMIT}{unsigned integer}]
Specifies the limit in bytes at which all memory allocations should fail if the
total allocated memory should increase beyond this.
\item[\optionpar{FAILFREQ}{unsigned integer}]
Specifies the frequency at which all memory allocations will randomly fail.
\item[\optionpar{FAILSEED}{unsigned integer}]
Specifies the random number seed which will be used when determining which
memory allocations will randomly fail.
\item[\optionpar{UNFREEDABORT}{unsigned integer}]
Specifies the minimum number of unfreed allocations at which to abort the
program just before program termination.
\end{description}

\vskip 12pt
\heading{Profiling}
\vskip 6pt

\begin{description}
\item[\option{PROF}]
Specifies that all memory allocations are to be profiled and sent to the
profiling output file.
\item[\optionpar{PROFFILE}{string}]
Specifies an alternative file in which to place all memory allocation profiling
information from the mpatrol library.
\item[\optionpar{AUTOSAVE}{unsigned integer}]
Specifies the frequency at which to periodically write the profiling data to the
profiling output file.
\item[\optionpar{SMALLBOUND}{unsigned integer}]
Specifies the limit in bytes up to which memory allocations should be classified
as small allocations for profiling purposes.
\item[\optionpar{MEDIUMBOUND}{unsigned integer}]
Specifies the limit in bytes up to which memory allocations should be classified
as medium allocations for profiling purposes.
\item[\optionpar{LARGEBOUND}{unsigned integer}]
Specifies the limit in bytes up to which memory allocations should be classified
as large allocations for profiling purposes.
\end{description}

\vskip 12pt

All of the function definitions in \texttt{mpatrol.h} can be disabled by
defining the \texttt{NDEBUG} preprocessor macro, which is the same macro used
to control the behaviour of the \function{assert} function.  If \texttt{NDEBUG}
is defined then no macro redefinition of functions will take place and all
special mpatrol library functions will evaluate to empty statements.  The
\texttt{mpalloc.h} header file will also be included in this case.  It is
intended that the \texttt{NDEBUG} preprocessor macro be defined in release
builds.

The \function{MP\_MALLOC} family of functions that are defined in
\texttt{mpalloc.h} are also defined in \texttt{mpatrol.h} when \texttt{NDEBUG}
is not defined.  The mpatrol versions of these functions contain more debugging
information than the mpalloc versions do, but they do not call the allocation
failure handler when no more memory is available (they cause the \error{OUTMEM}
error message to be given instead).

There may be problems during preprocessing when the preprocessor macros defining
the replacement C$++$ operators in \texttt{mpatrol.h} are used.  If this is the
case then either the \texttt{MP\_NOCPLUSPLUS} preprocessor macro can be defined
to disable all C$++$ support, or the \texttt{MP\_NONEWDELETE} preprocessor macro
can be defined to prevent the debugging versions of \texttt{operator new} and
\texttt{operator delete} from being used by default.  The preprocessor macros
\texttt{MP\_NEW}, \texttt{MP\_NEW\_NOTHROW} and \texttt{MP\_DELETE} will then
have to be used instead.

On systems that support it, global functions (with C linkage) in an executable
file or shared library whose names begin with \texttt{\_\_mp\_init\_} will be
noted when the mpatrol library first starts up and is reading the symbols.  Such
functions will then be called as soon as the mpatrol library is initialised,
which can be useful if the initialisation occurs before \function{main} is
called.  These functions must accept no arguments and must return no value.
Similar behaviour exists for global functions whose names begin with
\texttt{\_\_mp\_fini\_}, except that such functions will be executed when the
mpatrol library terminates.

\vskip 12pt
\heading{C dynamic memory allocation functions}
\vskip 6pt

\begin{description}
\item[\function{malloc}]
\hfill Allocates memory.
\item[\function{calloc}]
\hfill Allocates zero-filled memory.
\item[\function{memalign}]
\hfill Allocates memory with a specified alignment.
\item[\function{valloc}]
\hfill Allocates page-aligned memory.
\item[\function{pvalloc}]
\hfill Allocates a number of pages.
\item[\function{alloca}]
\hfill Allocates temporary memory.
\item[\function{strdup}]
\hfill Duplicates a string.
\item[\function{strndup}]
\hfill Duplicates a string with a maximum length.
\item[\function{strsave}]
\hfill Duplicates a string.
\item[\function{strnsave}]
\hfill Duplicates a string with a maximum length.
\item[\function{strdupa}]
\hfill Duplicates a string.
\item[\function{strndupa}]
\hfill Duplicates a string with a maximum length.
\item[\function{realloc}]
\hfill Resizes memory.
\item[\function{reallocf}]
\hfill Resizes memory and frees on failure.
\item[\function{recalloc}]
\hfill Resizes memory allocated by \function{calloc}.
\item[\function{expand}]
\hfill Resizes memory but does not relocate it.
\item[\function{free}]
\hfill Frees memory.
\item[\function{cfree}]
\hfill Frees memory allocated by \function{calloc}.
\item[\function{dealloca}]
\hfill Explicitly frees temporary memory.
\end{description}

\vskip 12pt
\heading{C dynamic memory extension functions}
\vskip 6pt

\begin{description}
\item[\function{xmalloc}]
\hfill Allocates memory without failure.
\item[\function{xcalloc}]
\hfill Allocates zero-filled memory without failure.
\item[\function{xstrdup}]
\hfill Duplicates a string without failure.
\item[\function{xrealloc}]
\hfill Resizes memory without failure.
\item[\function{xfree}]
\hfill Frees memory.
\end{description}

\vskip 12pt
\heading{C dynamic memory alternative functions}
\vskip 6pt

\begin{description}
\item[\function{MP\_MALLOC}]
\hfill Allocates memory without failure.
\item[\function{MP\_CALLOC}]
\hfill Allocates zero-filled memory without failure.
\item[\function{MP\_STRDUP}]
\hfill Duplicates a string without failure.
\item[\function{MP\_REALLOC}]
\hfill Resizes memory without failure.
\item[\function{MP\_FREE}]
\hfill Frees memory.
\item[\function{MP\_FAILURE}]
\hfill Sets the allocation failure handler.
\end{description}

\vskip 12pt
\heading{C$++$ dynamic memory allocation functions}
\vskip 6pt

\begin{description}
\item[\texttt{operator new}]
\hfill Allocates memory.
\item[\texttt{operator new[]}]
\hfill Allocates memory for an array.
\item[\texttt{operator delete}]
\hfill Frees memory.
\item[\texttt{operator delete[]}]
\hfill Frees memory allocated by \texttt{new[]}.
\item[\function{set\_new\_handler}]
\hfill Sets up an allocation failure handler.
\end{description}

\vskip 12pt
\heading{C memory operation functions}
\vskip 6pt

\begin{description}
\item[\function{memset}]
\hfill Fills memory with a specific byte.
\item[\function{bzero}]
\hfill Fills memory with the zero byte.
\item[\function{memccpy}]
\hfill Copies memory up to a specific byte.
\item[\function{memcpy}]
\hfill Copies non-overlapping memory.
\item[\function{memmove}]
\hfill Copies possibly-overlapping memory.
\item[\function{bcopy}]
\hfill Copies possibly-overlapping memory.
\item[\function{memcmp}]
\hfill Compares two blocks of memory.
\item[\function{bcmp}]
\hfill Compares two blocks of memory.
\item[\function{memchr}]
\hfill Searches memory for a specific byte.
\item[\function{memmem}]
\hfill Searches memory for specific bytes.
\end{description}

\vskip 12pt
\heading{mpatrol library functions}
\vskip 6pt

\begin{description}
\item[\function{\_\_mp\_atexit}]
\hfill Registers termination functions.
\item[\function{\_\_mp\_setoption}]
\hfill Sets an mpatrol library option.
\item[\function{\_\_mp\_getoption}]
\hfill Returns an mpatrol library option.
\item[\function{\_\_mp\_libversion}]
\hfill Returns the mpatrol library version.
\item[\function{\_\_mp\_strerror}]
\hfill Returns an error message string.
\item[\function{\_\_mp\_function}]
\hfill Returns an allocation type function name.
\item[\function{\_\_mp\_setuser}]
\hfill Sets the user data for an allocation.
\item[\function{\_\_mp\_setmark}]
\hfill Sets the marked flag for an allocation.
\item[\function{\_\_mp\_info}]
\hfill Returns information for an allocation.
\item[\function{\_\_mp\_syminfo}]
\hfill Returns symbol information for an address.
\item[\function{\_\_mp\_symbol}]
\hfill Returns symbol name for an address.
\item[\function{\_\_mp\_printinfo}]
\hfill Displays information for an allocation.
\item[\function{\_\_mp\_snapshot}]
\hfill Returns the current heap event number.
\item[\function{\_\_mp\_iterate}]
\hfill Iterates over allocations in the heap.
\item[\function{\_\_mp\_iterateall}]
\hfill Iterates over all allocations in the heap.
\item[\function{\_\_mp\_addallocentry}]
\hfill Adds an allocation to the leak table.
\item[\function{\_\_mp\_addfreeentry}]
\hfill Adds a deallocation to the leak table.
\item[\function{\_\_mp\_clearleaktable}]
\hfill Clears the leak table.
\item[\function{\_\_mp\_startleaktable}]
\hfill Starts automatic leak table entries.
\item[\function{\_\_mp\_stopleaktable}]
\hfill Stops automatic leak table entries.
\item[\function{\_\_mp\_leaktable}]
\hfill Displays the leak table.
\item[\function{\_\_mp\_memorymap}]
\hfill Displays a map of memory in the heap.
\item[\function{\_\_mp\_summary}]
\hfill Displays a summary of library statistics.
\item[\function{\_\_mp\_stats}]
\hfill Returns statistics about the heap.
\item[\function{\_\_mp\_check}]
\hfill Validates memory in the heap.
\item[\function{\_\_mp\_prologue}]
\hfill Sets up an allocation prologue handler.
\item[\function{\_\_mp\_epilogue}]
\hfill Sets up an allocation epilogue handler.
\item[\function{\_\_mp\_nomemory}]
\hfill Sets up an allocation failure handler.
\item[\function{\_\_mp\_printf}]
\hfill Writes user data to the log file.
\item[\function{\_\_mp\_vprintf}]
\hfill Writes user data to the log file.
\item[\function{\_\_mp\_locprintf}]
\hfill Logs user data and the location.
\item[\function{\_\_mp\_vlocprintf}]
\hfill Logs user data and the location.
\item[\function{\_\_mp\_logmemory}]
\hfill Displays a hex dump of memory.
\item[\function{\_\_mp\_logstack}]
\hfill Displays the current call stack.
\item[\function{\_\_mp\_logaddr}]
\hfill Displays information for an allocation.
\item[\function{\_\_mp\_edit}]
\hfill Invokes a text editor on a source file.
\item[\function{\_\_mp\_list}]
\hfill Lists a source file at a specific line.
\item[\function{\_\_mp\_view}]
\hfill Edits or lists a source file.
\item[\function{\_\_mp\_readcontents}]
\hfill Reads the contents of a memory allocation.
\item[\function{\_\_mp\_writecontents}]
\hfill Writes the contents of a memory allocation.
\item[\function{\_\_mp\_cmpcontents}]
\hfill Compares the contents of a memory allocation.
\item[\function{\_\_mp\_remcontents}]
\hfill Removes the contents of a memory allocation.
\end{description}

\vskip 12pt
\heading{mpatrol library variables}
\vskip 6pt

\begin{description}
\item[\variable{\_\_mp\_errno}]
\hfill Contains the most recent error code.
\end{description}

\vskip 12pt

The following table lists the warnings and errors that are likely to appear in
the mpatrol log file when problems with dynamic memory allocations and memory
operations occur.  Other types of warnings and errors may also appear in the log
file, but they are likely to be associated with parsing options and reading
symbols from executable files and so should be self-explanatory.

\vskip 12pt
\heading{Error abbreviation codes}
\vskip 6pt

\begin{description}
\item[\error{ALLOVF}]
\hfill Allocation has a corrupted overflow buffer.
\item[\error{ALLZER}]
\hfill Attempt to create an allocation of size 0.
\item[\error{BADALN}]
\hfill Alignment is not a power of two.
\item[\error{FRDCOR}]
\hfill Freed allocation has memory corruption.
\item[\error{FRDOPN}]
\hfill Attempt to perform operation on freed memory.
\item[\error{FRDOVF}]
\hfill Freed allocation has a corrupted overflow buffer.
\item[\error{FRECOR}]
\hfill Free memory corruption.
\item[\error{FREMRK}]
\hfill Attempt to free a marked memory allocation.
\item[\error{FRENUL}]
\hfill Attempt to free a \texttt{NULL} pointer.
\item[\error{FREOPN}]
\hfill Attempt to perform operation on free memory.
\item[\error{ILLMEM}]
\hfill Illegal memory access.
\item[\error{INCOMP}]
Attempt to resize or free memory allocated with an incompatible function.
\item[\error{MAXALN}]
\hfill Alignment is greater than the system page size.
\item[\error{MISMAT}]
Attempt to resize or free memory not pointing to the start of a memory
allocation.
\item[\error{NOTALL}]
\hfill Pointer has not been allocated.
\item[\error{NULOPN}]
\hfill Attempt to perform operation on a \texttt{NULL} pointer.
\item[\error{OUTMEM}]
\hfill Out of memory.
\item[\error{PRVFRD}]
Attempt to resize of free memory that has previously been freed.
\item[\error{RNGOVF}]
Attempt to perform a memory operation that overflows a memory allocation.
\item[\error{RNGOVL}]
Attempt to perform a non-overlapping memory operation that overlaps.
\item[\error{RSZNUL}]
\hfill Attempt to resize a \texttt{NULL} pointer.
\item[\error{RSZZER}]
\hfill Attempt to resize an allocation to size 0.
\item[\error{STROVF}]
Attempt to perform a string operation that overflows a memory allocation.
\item[\error{ZERALN}]
\hfill Alignment 0 is invalid.
\item[\error{ZEROPN}]
\hfill Attempt to perform a zero-sized operation.
\end{description}

\vskip 12pt

The commands that are distributed with the mpatrol library all parse their
command line options in a similar way to the UNIX \function{getopt} function.
Only the long names of the options are shown here.  If an option has a single
character equivalent it will be listed in the \flag{help} output.  Options that
accept numeric arguments can have their value specified in binary, octal,
decimal or hexadecimal notation.

\vskip 12pt
\heading{mpatrol command options}
\vskip 6pt

\begin{description}
\item[\flagpar{alloc-byte}{unsigned integer}]
\hfill See \option{ALLOCBYTE}.
\item[\flagpar{alloc-stop}{unsigned integer}]
\hfill See \option{ALLOCSTOP}.
\item[\flag{allow-oflow}]
\hfill See \option{ALLOWOFLOW}.
\item[\flagpar{auto-save}{unsigned integer}]
\hfill See \option{AUTOSAVE}.
\item[\flagpar{check}{unsigned range}]
\hfill See \option{CHECK}.
\item[\flag{check-all}]
\hfill See \option{CHECKALL}.
\item[\flag{check-allocs}]
\hfill See \option{CHECKALLOCS}.
\item[\flag{check-fork}]
\hfill See \option{CHECKFORK}.
\item[\flag{check-frees}]
\hfill See \option{CHECKFREES}.
\item[\flag{check-memory}]
\hfill See \option{CHECKMEMORY}.
\item[\flag{check-reallocs}]
\hfill See \option{CHECKREALLOCS}.
\item[\flagpar{def-align}{unsigned integer}]
\hfill See \option{DEFALIGN}.
\item[\flag{dynamic}]
Specifies that programs which were not linked with the mpatrol library should
also be traced, but only if they were dynamically linked.
\item[\flag{edit}]
\hfill See \option{EDIT}.
\item[\flagpar{fail-freq}{unsigned integer}]
\hfill See \option{FAILFREQ}.
\item[\flagpar{fail-seed}{unsigned integer}]
\hfill See \option{FAILSEED}.
\item[\flagpar{free-byte}{unsigned integer}]
\hfill See \option{FREEBYTE}.
\item[\flagpar{free-stop}{unsigned integer}]
\hfill See \option{FREESTOP}.
\item[\flag{help}]
Displays a quick-reference option summary.
\item[\flagpar{large-bound}{unsigned integer}]
\hfill See \option{LARGEBOUND}.
\item[\flag{leak-table}]
\hfill See \option{LEAKTABLE}.
\item[\flagpar{limit}{unsigned integer}]
\hfill See \option{LIMIT}.
\item[\flag{list}]
\hfill See \option{LIST}.
\item[\flag{log-all}]
\hfill See \option{LOGALL}.
\item[\flag{log-allocs}]
\hfill See \option{LOGALLOCS}.
\item[\flagpar{log-file}{string}]
\hfill See \option{LOGFILE}.
\item[\flag{log-frees}]
\hfill See \option{LOGFREES}.
\item[\flag{log-memory}]
\hfill See \option{LOGMEMORY}.
\item[\flag{log-reallocs}]
\hfill See \option{LOGREALLOCS}.
\item[\flagpar{medium-bound}{unsigned integer}]
\hfill See \option{MEDIUMBOUND}.
\item[\flagpar{no-free}{unsigned integer}]
\hfill See \option{NOFREE}.
\item[\flag{no-protect}]
\hfill See \option{NOPROTECT}.
\item[\flagpar{oflow-byte}{unsigned integer}]
\hfill See \option{OFLOWBYTE}.
\item[\flagpar{oflow-size}{unsigned integer}]
\hfill See \option{OFLOWSIZE}.
\item[\flag{oflow-watch}]
\hfill See \option{OFLOWWATCH}.
\item[\flag{page-alloc-lower}]
\hfill See \option{PAGEALLOC=LOWER}.
\item[\flag{page-alloc-upper}]
\hfill See \option{PAGEALLOC=UPPER}.
\item[\flag{preserve}]
\hfill See \option{PRESERVE}.
\item[\flag{prof}]
\hfill See \option{PROF}.
\item[\flagpar{prof-file}{string}]
\hfill See \option{PROFFILE}.
\item[\flagpar{prog-file}{string}]
\hfill See \option{PROGFILE}.
\item[\flag{read-env}]
Reads and passes through the contents of the \texttt{MPATROL\_OPTIONS}
environment variable.
\item[\flagpar{realloc-stop}{unsigned integer}]
\hfill See \option{REALLOCSTOP}.
\item[\flag{safe-signals}]
\hfill See \option{SAFESIGNALS}.
\item[\flag{show-all}]
\hfill See \option{SHOWALL}.
\item[\flag{show-env}]
Displays the contents of the \texttt{MPATROL\_OPTIONS} environment variable.
\item[\flag{show-free}]
\hfill See \option{SHOWFREE}.
\item[\flag{show-freed}]
\hfill See \option{SHOWFREED}.
\item[\flag{show-map}]
\hfill See \option{SHOWMAP}.
\item[\flag{show-symbols}]
\hfill See \option{SHOWSYMBOLS}.
\item[\flag{show-unfreed}]
\hfill See \option{SHOWUNFREED}.
\item[\flagpar{small-bound}{unsigned integer}]
\hfill See \option{SMALLBOUND}.
\item[\flag{threads}]
Specifies that the program to be run is multithreaded if the \flag{dynamic}
option is used.
\item[\flag{trace}]
\hfill See \option{TRACE}.
\item[\flagpar{trace-file}{string}]
\hfill See \option{TRACEFILE}.
\item[\flagpar{unfreed-abort}{unsigned integer}]
\hfill See \option{UNFREEDABORT}.
\item[\flag{use-debug}]
\hfill See \option{USEDEBUG}.
\item[\flag{use-mmap}]
\hfill See \option{USEMMAP}.
\item[\flag{version}]
Displays the version number of the \command{mpatrol} command.
\end{description}

\vskip 12pt
\heading{mprof command options}
\vskip 6pt

\begin{description}
\item[\flag{addresses}]
Specifies that different call sites from within the same function are to be
differentiated and that the names of all functions should be displayed with
their call site offset in bytes.
\item[\flag{call-graph}]
Specifies that the allocation call graph should be displayed.
\item[\flag{counts}]
Specifies that certain tables should be sorted by the number of allocations or
deallocations rather than the total number of bytes allocated or deallocated.
\item[\flagpar{graph-file}{file}]
Specifies that the allocation call graph should also be written to a graph
specification file for later visualisation with \command{dot}.
\item[\flag{help}]
Displays a quick-reference option summary.
\item[\flag{leaks}]
Specifies that memory leaks rather than memory allocations are to be written to
the graph specification file.
\item[\flagpar{stack-depth}{depth}]
Specifies the maximum stack depth to use when calculating if one call site has
the same call stack as another call site.  This also specifies the maximum
number of functions to display in a call stack.
\item[\flag{version}]
Displays the version number of the \command{mprof} command.
\end{description}

\vskip 12pt
\heading{mptrace command options}
\vskip 6pt

The \command{mptrace} command can be built with GUI support on certain
platforms.  Any of the following options that are marked as being specific to
the GUI version of \command{mptrace} will be read by the X command line parser
rather than directly by \command{mptrace}.  As a result they are parsed
according to X toolkit rules and do not appear in the quick-reference option
summary produced by the \flag{--help} option.  The application class for setting
\command{mptrace} X resources is called \texttt{MPTrace}.

\begin{description}
\item[\flagpar{alloc}{colour}]
Specifies the colour to use for displaying allocated memory (GUI only).
\item[\flagpar{base}{address}]
Specifies the base address of the visible address space displayed in the memory
map (GUI only).
\item[\flagpar{delay}{length}]
Specifies that a small delay of a certain length should be added after drawing
each memory allocation event (GUI only).
\item[\flagpar{free}{colour}]
Specifies the colour to use for displaying free memory (GUI only).
\item[\flag{gui}]
Displays the GUI (if supported).
\item[\flagpar{height}{size}]
Specifies the height (in pixels) of the drawing area (GUI only).
\item[\flagpar{hatf-file}{file}]
Specifies that the trace should also be written to a file in Heap Allocation
Trace Format (HATF).
\item[\flag{help}]
Displays a quick-reference option summary.
\item[\flagpar{internal}{colour}]
Specifies the colour to use for displaying internal heap memory (GUI only).
\item[\flagpar{sim-file}{file}]
Specifies that a trace-driven memory allocation simulation program written in C
should be written to a file.
\item[\flag{source}]
Displays source-level information for each event in the tracing table, if
available.
\item[\flagpar{space}{size}]
Specifies the size (in megabytes) of the visible address space displayed in the
memory map (GUI only).
\item[\flagpar{unalloc}{colour}]
Specifies the colour to use for displaying unallocated heap memory (GUI only).
\item[\flagpar{view-height}{size}]
Specifies the height (in pixels) of the window (GUI only).
\item[\flagpar{view-width}{size}]
Specifies the width (in pixels) of the window (GUI only).
\item[\flagpar{width}{size}]
Specifies the width (in pixels) of the drawing area (GUI only).
\item[\flag{verbose}]
Specifies that the tracing table should be displayed.
\item[\flag{version}]
Displays the version number of the \command{mptrace} command.
\end{description}

\vskip 12pt
\heading{mleak command options}
\vskip 6pt

\begin{description}
\item[\flag{help}]
Displays a quick-reference option summary.
\item[\flag{ignore}]
Specifies that the list of unfreed allocations in the log file should be
ignored.
\item[\flagpar{max-stack}{depth}]
Specifies the maximum stack depth to display.
\item[\flag{version}]
Displays the version number of the \command{mleak} command.
\end{description}

\vskip 12pt
\heading{mpsym command options}
\vskip 6pt

\begin{description}
\item[\flag{help}]
Displays a quick-reference option summary.
\item[\flag{skip}]
Skip symbols marked as \texttt{???} in the log file.
\item[\flag{version}]
Displays the version number of the \command{mpsym} command.
\end{description}

\vskip 12pt
\heading{mpedit command options}
\vskip 6pt

The \command{mpedit} command recognises the \texttt{EDITOR} environment variable
which can be used to specify the text editor that it should use to edit source
files.  It also recognises the \texttt{MPATROL\_SOURCEPATH} environment variable
which can be used to specify a colon-separated list of directories that should
be used to help search for source files.

\begin{description}
\item[\flagpar{editor}{filename}]
Specifies the text editor to use.
\item[\flag{help}]
Displays a quick-reference option summary.
\item[\flag{listing}]
Displays a context listing of the source line instead of invoking the text
editor.
\item[\flagpar{source-dir}{directory}]
Adds a directory to the search path used to locate the source file.
\item[\flag{version}]
Displays the version number of the \command{mpedit} command.
\end{description}

\vskip 12pt
\heading{hexwords command options}
\vskip 6pt

\begin{description}
\item[\flag{help}]
Displays a quick-reference option summary.
\item[\flagpar{match}{exact\texttt{|}lower\texttt{|}upper\texttt{|}any}]
Sets the type of case-sensitivity to use.
\item[\flagpar{maximum}{count}]
Sets the maximum number of letters to match.
\item[\flagpar{minimum}{count}]
Sets the minimum number of letters to match.
\item[\flag{version}]
Displays the version number of the \command{hexwords} command.
\end{description}

\vskip 12pt
Copyright \copyright 1997-2008 Graeme S. Roy.
\vskip 6pt

This reference card may be freely distributed under the terms of the GNU General
Public License.

\end{multicols}

\end{document}
